What is Power Electronics?

\begin{define}
\textbf{Source:} something that generates power

\textbf{Load:} something that consumes power

\textbf{Power electronics:} application of electronics and circuitry to control the conversion of one form to another
\end{define}

Converter types between AC and DC Power: DC stands for "direct current" and can be visualized as a constant voltage over time. One example is a battery and photovoltaic panel. AC power stands for "alternating current" and is a sinusoidal voltage in time. An example of this is the power from the outlet. There are four basic types of converters.
\begin{itemize}
    \item AC-DC: AC source to DC load, which is commonly called a rectifier like in the use of a laptop charger
    \item DC-DC: DC source to DC load, battery pack USB
    \item DC-AC: DC source to AC load, also commonly called an inverter like for a photovoltaic to grid system
    \item AC-AC: AC source to AC load, not as common but used in wond power system
\end{itemize}

\subsection{Average and Root Mean Square (RMS) Calculations}
Period waveforms repeat their shape across each period. The average value of a sine wave is 0. $\langle v(t) \rangle = \frac{1}{T} \int_0^T v(t) \,dt$ will represent the average value here.

The RMS is represented as capital V listed below
\begin{define}
    \[V = \sqrt{\frac{1}{T} \int_0^T v(t)^2 \,dt}\]
\end{define}
We can think of the RMS value as the equivalent voltage if we put the waveform of choice across a resistor

The power is equal to the voltage waveform squared over the resistor. The triangle brackets represent the average here.
\begin{define}
    \[\langle P \rangle = \langle \frac{v^2}{R} \rangle = v^2\]
\end{define}
Remember that 
\[\langle v \rangle ^2 \neq v^2\]

If we take a look at this sine wave and say that this represents current and connected this to a resistor, average power would not be 0 even though average current is 0. $I(t)$ here represents the instantaneous value. The resistor generates (consumes) power at both the negative and positive parts of this waveform.
\begin{center}
    \begin{tikzpicture}
        % Axes
        \draw[->] (0,0) -- (6.5,0) node[right] {$t$};
        \draw[->] (0,-1.5) -- (0,1.5) node[above] {$I(t)$};
        
        % Sine wave
        \draw[domain=0:2*pi,smooth,variable=\x,blue] plot ({\x},{sin(\x r)});
        
        % Labeling
        \draw[dashed] (pi/2,0) -- (pi/2,1) node[pos=0.5, right] {$\frac{\pi}{2}$};
        \draw[dashed] (3*pi/2,0) -- (3*pi/2,-1) node[pos=0.5, left] {$\frac{3\pi}{2}$};
        
        % Axes labels
        \node[below] at (pi,0) {$\pi$};
        \node[below] at (2*pi,0) {$2\pi$};
        \node[left] at (0,1) {1};
        \node[left] at (0,-1) {-1};
    \end{tikzpicture}
\end{center}

\begin{sanity}
    Is the RMS value always greater than or equal to the average value? Yes. Recall the definitions of average value and the RMS value above. My intuition behind this is that you're squaring a periodic function there will be no negative values.
\end{sanity}

Sine wave RMS value calculations:

Define $x(t) = X_{peak} \sin{(t \times \frac{2\pi}{T})}$. We multiply $t$ by a factor of $\frac{2\pi}{T}$ since we are working in units of time.
\begin{equation} 
    X_{RMS} = \sqrt{\frac{1}{T} \int_0^T X_{peak}^2 \sin^2{(t \times \frac{2\pi}{T})} \,dt} \tag{1} \\
    hi
\end{equation}
